\section{Previous work}

\subsection{\emacs{}}

Highlighting is based on string matching, and no attempt is made to
determine the symbols that are present in the current package.
Even when the current package does not use the \texttt{common-lisp}
package, strings that match \commonlisp{} symbols are highlighted
nevertheless.

Indentation is also based on string matching, resulting in the text
being indented as \commonlisp{} code even when it is not.
Furthermore, indentation does not take into account the role of a
symbol in the code.  So, for example, if a lexical variable named
(say) \texttt{prog1} is introduced in a \texttt{let} binding and
it is followed by a newline, the following line is indented as if the
symbol \texttt{prog1} were the name of a \commonlisp{} function as
opposed to the name of a lexical variable.

\subsection{\climacs{}}

Like \emacs{}, \climacs{} does not take the current package into
account.  The parser is based on table-driven parsing techniques
such as LALR parsing, except that the parsing table was derived
manually.  The parser is incremental in that the state of the parser
is saved for certain points in the buffer, so that parsing can be
restarted from such a point without requiring the entire buffer to be
parsed from the beginning.

Unlike \emacs{}, the \climacs{} parser is more accurate when it comes
to the role of symbols in the program code.  In many cases, it is able
to distinguish between a symbol used as the name of a function and the
same symbol used as a lexical variable.
